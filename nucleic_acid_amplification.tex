\documentclass{article}

\oddsidemargin = -0.25in
\headheight = 12pt
\textheight = 9in
\hoffset = 0pt
\topmargin = -0.5in
\headsep = 25pt
\textwidth = 7in

\title{RT-LAMP assay for detecting lentiviruses}

\begin{document}
\maketitle
%\tableofcontents
%\newpage

\section{Introduction}

Tests for known lentivirus retroviruses with a high viral diversity such as HIV are very specific and could miss divergent HIV strains \cite{bartolo2012hiv}\cite{luft2011hiv}. In \cite{voisset2008human}, a number of studies designing degenerate PCR primers for detecting lentiviruses are reviewed. In \cite{giovine1994absence}, PCR primers targetting a conserved region of the pol gene across five different lentivirus sequences were designed. The goal of the study was to look for evidence of a lentivirus in patients with rhematoid arthritis. 


\section{Design}

design constraints balance minimal design/development time, minimal equipment requirements, adequate performance.

\subsection{Sample Collection}

minimal sample volume, blood drop from lanclet 

\subsection{Lysis}

\cite{curtis2008rapid}

\subsection{Reaction}



\bibliographystyle{ieeetr}
\bibliography{nucleic_acid_amplification}


\end{document}
